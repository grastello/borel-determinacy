\documentclass[aspectratio=43]{beamer}
\usepackage[italian]{babel}
\usepackage[utf8]{inputenc}

\usetheme{Dresden}
\usecolortheme{beaver}

\usepackage{libertine}
\usepackage{unicode-math}

\usetheme{default}

\usepackage{mathtools}
\usepackage{amsmath}
\usepackage{amsfonts}
\usepackage{amssymb}
\usepackage{amsthm}

\usepackage{tikz}

\begin{document}

\section{Giochi di Gale-Stewart}

\begin{frame}{Giochi di Gale-Stewart su \(A\)}
  Sia \(A \neq \emptyset\) un insieme e \(X \subseteq A^\omega\) un \textbf{payoff set}.
  Consideriamo il gioco di Gale-Stewart \(G(A, X)\)
  \begin{equation*}
    \begin{matrix}
      \mathrm{I} & a_0 & & a_2 & & \ldots\\
      \mathrm{II} & & a_1 & & a_3 & \ldots
    \end{matrix}
  \end{equation*}
  \pause
  dove
  \begin{itemize}
  \item \(\mathrm{I}\) vince se \((a_n)_{n < \omega} \in X\);
  \item \(\mathrm{II}\) vince se \((a_n)_{n < \omega} \not\in X\).
  \end{itemize}
\end{frame}

\begin{frame}{Giochi di Gale-Stewart su \(T\)}
  Sia \(T \subseteq A^{<\omega}\) e \(X\subseteq [T]\) un payoff set; allora possiamo considerare il gioco \(G(T, X)\)
  \begin{equation*}
    \begin{matrix}
      \mathrm{I} & a_0 & & a_2 & & \ldots\\
      \mathrm{II} & & a_1 & & a_3 & \ldots
    \end{matrix}
  \end{equation*}
  con la restrizione aggiunta che \((a_0, \ldots, a_n) \in T\) per ogni \(n < \omega\) e le medesime condizioni di vittoria.
  \pause
  \begin{block}{Remark}
    Se \(T = A^{<\omega}\) otteniamo i giochi di Gale-Stewart su \(A\).
  \end{block}
\end{frame}

\section{Strategie}

\begin{frame}{Strategie}
  Fissiamo un gioco \(G(X, T)\).
  \begin{block}{Definizione}
     Una \textbf{strategia} per \(\mathrm{I}\) è un albero \(\sigma \subseteq T\) tale che
    \begin{itemize}
    \item[1.] \(\sigma\) è potato e non vuoto;
    \item[2.] se \((a_0, \ldots, a_{2j}) \in \sigma\) allora ogni \((a_0, \ldots, a_{2j}, a_{2j+1}) \in T\) è in \(\sigma\);
    \item[3.] se \((a_0, \ldots, a_{2j - 1}) \in \sigma\) allora esiste un unico \(a_{2j} \in A\) tale che \((a_0, \ldots, a_{2j - 1}, a_{2j}) \in \sigma\).
    \end{itemize}
  \end{block}
\end{frame}

\begin{frame}{Strategie}
  Se \(A = \{0, 1, 2\}\) e \(T = A^{<\omega}\) allora
  \begin{equation*}
    \begin{tikzpicture}
      \node (1) at (0, 0){\(0\)};
      \node (11) at (-2, -1){\(0\)};
      \node (12) at (0, -1){\(1\)};
      \node (13) at (2, -1){\(2\)};
      \node (21) at (-2, -2){\(0\)};
      \node (22) at (0, -2){\(1\)};
      \node (23) at (2, -2){\(2\)};
      \node (dots) at (0, -3.5){\(\vdots\)};
      \node (dots) at (-2, -3.5){\(\vdots\)};
      \node (dots) at (2, -3.5){\(\vdots\)};

      \node (a) at (3.5, 0){\(\mathrm{I}\)};
      \node (a) at (3.5, -1){\(\mathrm{II}\)};
      \node (a) at (3.5, -2){\(\mathrm{I}\)};

      \draw[-] (1) -- (11) -- (21);
      \draw[-] (1) -- (12) -- (22);
      \draw[-] (1) -- (13) -- (23);

      \draw[-] (21) -- (-2.5, -3);
      \draw[-] (21) -- (-2, -3);
      \draw[-] (21) -- (-1.5, -3);

      \draw[-] (22) -- (-0.5, -3);
      \draw[-] (22) -- (0, -3);
      \draw[-] (22) -- (0.5, -3);

      \draw[-] (23) -- (1.5, -3);
      \draw[-] (23) -- (2, -3);
      \draw[-] (23) -- (2.5, -3);
    \end{tikzpicture}
  \end{equation*}
  è una strategia per \(\mathrm{I}\).
\end{frame}

\begin{frame}{Strategie}
  \begin{block}{Definizione}
    Una strategia \(\sigma \subseteq T\) per \(\mathrm{I}\) è \textbf{vincente} se \([\sigma] \subseteq X\) i.e. se \(\mathrm{I}\) vince ogni partita giocata seguendo \(\sigma\).
  \end{block}
  \pause
  Similmente definiamo strategie per \(\mathrm{II}\).
  \pause
  \begin{block}{Remark}
    Siccome \(G(X, T)\) non può finire in un pareggio non è possibile che sia \(\mathrm{I}\) che \(\mathrm{II}\) abbiano una strategia vincente.
  \end{block}
\end{frame}

\section{Determinatezza dei Giochi Chiusi}

\begin{frame}{Determinatezza}
  \begin{block}{Definizione}
    Un gioco \(G(X, T)\), o solamente l'insieme \(X \subseteq T\), si dice \textbf{determinato} se uno dei due giocatori ha una strategia vincente.
  \end{block}
  \pause
  \begin{block}{Domande}
    \begin{itemize}
    \item I chiusi e gli aperti sono determinati?
    \item I Boreliani sono determinati?
    \item Gli analitici sono determinati?
    \end{itemize}
  \end{block}
\end{frame}

\begin{frame}{Determinatezza dei giochi chiusi}
  \begin{block}{Teorema (Gale-Stewart)}
    Dato \(T \subseteq A^{<\omega}\) potato e non-vuoto se \(X \subseteq [T]\) è aperto (o chiuso) in \([T]\) allora \(G(X, T)\) è determinato.
  \end{block}
\end{frame}

\begin{frame}{Posizioni non perdenti}
    \begin{block}{Definizione}
    Data una posizione \(p = (a_0, \ldots, a_{2n+1}) \in T\) diciamo che \(p\) è \textbf{non perdente} per \(\mathrm{I}\) se \(\mathrm{II}\) non ha una strategia vincente a partire da \(p\).
    Formalmente \(p\) è non perdente per \(\mathrm{I}\) se \(\mathrm{II}\) non ha una strategia vincente per il gioco \(G(T_p, X_p)\)     dove
    \begin{equation*}
      T_p = \{ s \in A^{<\omega} : p^\smallfrown s \in T\} \quad \text{e} \quad X_p = \{x \in A^\omega : p^\smallfrown x \in X\}.
    \end{equation*}
  \end{block}
  \pause
  \begin{block}{Remark}
    Se una posizione \(p = (a_0, \ldots, a_{2n+1}) \in T\) è non perdente per \(\mathrm{I}\) allora esiste un \(a_{2n + 2}\) che \(\mathrm{I}\) può giocare (i.e. \((a_{2n +2}) \in T_p\)) tale che per ogni \(a_{2n + 3}\) con cui \(\mathrm{II}\) può rispondere (i.e. \((a_{2n + 2}, a_{2n + 3}) \in T_p\)) la posizione \(p^\smallfrown (a_{2n + 2}, a_{2n + 3}) \in T\) sia ancora non perdente per \(\mathrm{I}\).
  \end{block}
\end{frame}

\begin{frame}{Dimostrazione}
  \begin{block}{}
    Lavoriamo con \(X\) chiuso ed assumiamo che \(\mathrm{II}\) non abbia strategia vincente (se la ha allora abbiamo il teorema).
  \end{block}
  \pause

  \begin{block}{}
    Per costruire una strategia vincente per \(\mathrm{I}\) osserviamo che se \(\mathrm{II}\) non ha una strategia vincente allora \(\emptyset\) è una posizione non perdente per \(\mathrm{I}\).
    Allora \(\mathrm{I}\) può, come prima mossa, giocare un \(a_0\) tale che per ogni \(a_1\) per cui \((a_0, a_1) \in T\) quest'ultima posizione è ancora non perdente per \(\mathrm{I}\).
  \end{block}
  \pause

  \begin{block}{}
    Adesso per ogni \(a_1\) con cui \(\mathrm{II}\) può rispondere, per scelta di \(a_0\), esiste un \(a_2\) che \(\mathrm{I}\) può giocare e tale che per ogni \(a_3\) tale che \((a_0, a_1, a_2, a_3) \in T\) questa sia una posizione non perdente per \(\mathrm{I}\).
  \end{block}
\end{frame}

\begin{frame}{Dimostrazione}
  \begin{block}{}
    In questo modo costruiamo una strategia \(\sigma \subseteq T\) per \(\mathrm{I}\) con la proprietà che, se \((a_0, \ldots, a_{2n+1}) \in \sigma\) allora questa è una posizione non perdente per \(\mathrm{I}\).
  \end{block}
  \pause
  \begin{block}{}
    Sia \((a_n)_n\) una partita dove \(\mathrm{I}\) ha seguito \(\sigma\) i.e. \((a_n)_n \in [\sigma]\).
    Se \((a_n)_n \not\in X\), cioè \((a_n)_n \in [T] - X\), siccome \([T]\) è chiuso abbiamo che esiste un \(k < \omega\) tale che
    \begin{equation*}
      N_{(a_0, \ldots, a_{2k+1})}\cap[T] \subseteq [T] - X.
    \end{equation*}
    Ma, se esiste un tale \(k\), abbiamo che \((a_0, \ldots, a_{2k+1})\) è una posizione perdente per \(\mathrm{I}\) siccome \(\mathrm{II}\) vince giocando mosse arbitrarie.

  \end{block}
\end{frame}

\begin{frame}{Dimostrazione}
  \begin{block}{}
    Questo è assurdo perché \((a_0, \ldots, a_{2k+1}) \in \sigma\) e dunque deve essere non perdente per \(\mathrm{I}\).
    Dobbiamo dunque avere \((a_n)_n \in X\) e quindi \([\sigma] \subseteq X\); \(\mathrm{I}\) ha una strategia vincente.
  \end{block}
  \pause
  \begin{block}{}
    Se \(X\) è aperto possiamo ripetere lo stesso argomento invertendo i ruoli di \(\mathrm{I}\) e \(\mathrm{II}\).
    \begin{flushright}
      \qedsymbol
    \end{flushright}
  \end{block}
\end{frame}

\section{Ricoprimenti}

\begin{frame}{Ricoprimenti}
  \begin{block}{Definizione}
    Dato \(T \subseteq A^{<\omega}\) potato e non-vuoto.
    Un \textbf{ricoprimento} di \(T\) è una tripla \((\tilde{T}, \pi, \varphi)\) dove
    \begin{itemize}
    \item[1.] \(\tilde{T}\) è un albero potato e non-vuoto;
    \item[2.] \(\pi:\tilde{T} \to T\) è monotona i.e. se \(s \subseteq t\) allora \(\pi(s) \subseteq \pi(t)\) e  tale che \(|\pi(s)| = |s|\);
    \item[3.] \(\varphi\) mappa strategie per \(\mathrm{I}\) (e per \(\mathrm{II}\)) in \(\tilde{T}\) a strategie per \(\mathrm{I}\) (e per \(\mathrm{II}\)) in \(T\) in modo tale che \(\varphi(\tilde{\sigma})\) ristretta a posizioni di lunghezza \(\leq n\) dipende solo da \(\tilde{\sigma}\) ristretta a posizioni di lunghezza \(\leq n\);
    \item[4.] se \(\tilde{\sigma}\) è una strategia per \(\mathrm{I}\) (o \(\mathrm{II}\)) in \(\tilde{T}\) e \(x \in [\varphi(\tilde{\sigma})] \subseteq [T]\) allora esiste \(\tilde{x} \in [\tilde{\sigma}] \subseteq [\tilde{T}]\) tale che \(\pi(\tilde{x}) = x\).
    \end{itemize}
  \end{block}
\end{frame}

\begin{frame}
  \begin{block}{}
    Se \((\tilde{T}, \pi, \varphi)\) è un ricoprimento di \(T\) e \(X \subseteq [T]\) allora a \(G(T, X)\) associamo \(G(\tilde{T}, \tilde{X})\) dove \(\tilde{X} = \pi^{-1}(X)\).
    Infatti ad ogni partita \(\tilde{x} \in [\tilde{T}]\) corrisponde \(\pi(x) \in [T]\), una partita di \(G(T, X)\).
  \end{block}
  \pause
  \begin{block}{Osservazione}
    Se \(\tilde{\sigma}\) è una strategia vincente per \(\mathrm{I}\) (o \(\mathrm{II}\)) in \(G(\tilde{T}, \tilde{X})\) allora \(\varphi(\tilde{\sigma})\) è una strategia vincente per \(\mathrm{I}\) (o \(\mathrm{II}\)) in \(G(T, X)\).
  \end{block}
  \pause
  \begin{block}{}
    Se così non fosse dovrebbe esserci un \(x \in [\varphi(\tilde{\sigma})]\) tale che \(x \not\in X\).
    Ma, per la condizione 4, esiste un \(\tilde{x} \in [\tilde{\sigma}]\) tale che \(\pi(\tilde{x}) = x\).
    Ora siccome \(\tilde{\sigma}\) è vincente \(\tilde{x} \in \tilde{X}\) e dunque \(\pi(\tilde{x}) = x \in X\); assurdo.
  \end{block}
\end{frame}

\begin{frame}{\(k\)-ricoprimenti}
  \begin{block}{Definizione}
    Fissato \(k < \omega\) un ricoprimento \((\tilde{T}, \pi, \varphi)\) è un \textbf{\(k\)-ricoprimento} se
    \begin{itemize}
    \item[1.] \(T \upharpoonright 2k = \tilde{T} \upharpoonright 2k\) dove \(T \upharpoonright n = \{x \in T : |x| \leq n\}\);
    \item[2.] \(\pi \upharpoonright (\tilde{T} \upharpoonright 2k)\) è la funzione identità.
    \end{itemize}
  \end{block}
  \pause
  \begin{block}{}
    Questo significa che nel gioco \(G(\tilde{T}, \tilde{X})\) le prime \(k\) mosse di entrambi i giocatori sono le stesse che in \(G(T, X)\).
  \end{block}
  \pause
  \begin{block}{}
    Inoltre se \(\tilde{\sigma}\) è una strategia per \(\tilde{T}\) allora \(\varphi(\tilde{\sigma}) \upharpoonright 2k = \tilde{\sigma} \upharpoonright 2k\) i.e. \(\varphi\) mappa strategie in strategie senza cambiare le prime \(k\) mosse.
  \end{block}
\end{frame}

\begin{frame}
  \begin{block}{Definizione}
    Un ricoprimento \((\tilde{T}, \pi, \varphi)\) \textbf{srotola} \(X \subseteq [T]\) se \(\pi^{-1}(X)\) è un clopen di \(\tilde{T}\).
  \end{block}
  \pause
  \begin{block}{Osservazione}
    In particolare se \((\tilde{T}, \pi, \varphi)\) srotola \(X\) allora \(G(\tilde{T}, \tilde{X})\) è un gioco chiuso, dunque determinato, e quindi anche \(G(T, X)\) è determinato perché \(\varphi\) trasforma strategie vincenti in strategie vincenti.
  \end{block}
\end{frame}

\section{Determinatezza Boreliana}

\begin{frame}{Determinatezza Boreli}
  \begin{block}{Teorema (Determinatezza Boreliana)}
    Dato \(T \subseteq A^{<\omega}\) potato e non-vuoto se \(X \subseteq [T]\) è Boreliano in \([T]\) allora \(G(X, T)\) è determinato.
  \end{block}
  \pause
  \begin{block}{Teorema}
    Dato \(T \subseteq A^{<\omega}\) potato e non-vuoto se \(X \subseteq [T]\) è Boreliano in \([T]\) allora per ogni \(k < \omega\) esiste un \(k\)-ricoprimento di \(T\) che srotola \(X\).
  \end{block}
\end{frame}

\begin{frame}
  \begin{block}{Lemma 1}
    
  \end{block}
\end{frame}

\end{document}