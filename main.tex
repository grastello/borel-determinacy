\documentclass[aspectratio=43]{beamer}
\usepackage[italian]{babel}
\usepackage[utf8]{inputenc}

\usetheme{Dresden}
\usecolortheme{beaver}

\usepackage{libertine}
\usepackage{unicode-math}

\usetheme{default}

\usepackage{mathtools}
\usepackage{amsmath}
\usepackage{amsfonts}
\usepackage{amssymb}
\usepackage{amsthm}

\begin{document}

\section{Giochi di Gale-Stewart}

\begin{frame}{Giochi di Gale-Stewart su \(A\)}
  Sia \(A \neq \emptyset\) un insieme e \(X \subseteq A^\omega\) un \textbf{payoff set}.
  Consideriamo il gioco di Gale-Stewart \(G(A, X)\)
  \begin{equation*}
    \begin{matrix}
      \mathrm{I} & a_0 & & a_2 & & \ldots\\
      \mathrm{II} & & a_1 & & a_3 & \ldots
    \end{matrix}
  \end{equation*}
  \pause
  dove
  \begin{itemize}
  \item \(\mathrm{I}\) vince se \((a_n)_{n < \omega} \in X\);
  \item \(\mathrm{II}\) vince se \((a_n)_{n < \omega} \not\in X\).
  \end{itemize}
\end{frame}

\begin{frame}{Giochi di Gale-Stewart su \(T\)}
  Sia \(T \subseteq A^{<\omega}\) e \(X\subseteq [T]\) un payoff set; allora possiamo considerare il gioco \(G(T, X)\)
  \begin{equation*}
    \begin{matrix}
      \mathrm{I} & a_0 & & a_2 & & \ldots\\
      \mathrm{II} & & a_1 & & a_3 & \ldots
    \end{matrix}
  \end{equation*}
  con la restrizione aggiunta che \((a_0, \ldots, a_n) \in T\) per ogni \(n < \omega\) e le medesime condizioni di vittoria.
  \pause
  \begin{block}{Remark}
    Se \(T = A^{<\omega}\) otteniamo i giochi di Gale-Stewart su \(A\).
  \end{block}
\end{frame}
\end{document}