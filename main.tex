\documentclass[aspectratio=43]{beamer}
\usepackage[italian]{babel}
\usepackage[utf8]{inputenc}

\usetheme{Dresden}
\usecolortheme{beaver}

\usepackage{libertine}
\usepackage{unicode-math}

\usetheme{default}

\usepackage{mathtools}
\usepackage{amsmath}
\usepackage{amsfonts}
\usepackage{amssymb}
\usepackage{amsthm}

\usepackage{tikz}

\begin{document}

\section{Giochi di Gale-Stewart}

\begin{frame}{Giochi di Gale-Stewart su \(A\)}
  Sia \(A \neq \emptyset\) un insieme e \(X \subseteq A^\omega\) un \textbf{payoff set}.
  Consideriamo il gioco di Gale-Stewart \(G(A, X)\)
  \begin{equation*}
    \begin{matrix}
      \mathrm{I} & a_0 & & a_2 & & \ldots\\
      \mathrm{II} & & a_1 & & a_3 & \ldots
    \end{matrix}
  \end{equation*}
  \pause
  dove
  \begin{itemize}
  \item \(\mathrm{I}\) vince se \((a_n)_{n < \omega} \in X\);
  \item \(\mathrm{II}\) vince se \((a_n)_{n < \omega} \not\in X\).
  \end{itemize}
\end{frame}

\begin{frame}{Giochi di Gale-Stewart su \(T\)}
  Sia \(T \subseteq A^{<\omega}\) e \(X\subseteq [T]\) un payoff set; allora possiamo considerare il gioco \(G(T, X)\)
  \begin{equation*}
    \begin{matrix}
      \mathrm{I} & a_0 & & a_2 & & \ldots\\
      \mathrm{II} & & a_1 & & a_3 & \ldots
    \end{matrix}
  \end{equation*}
  con la restrizione aggiunta che \((a_0, \ldots, a_n) \in T\) per ogni \(n < \omega\) e le medesime condizioni di vittoria.
  \pause
  \begin{block}{Remark}
    Se \(T = A^{<\omega}\) otteniamo i giochi di Gale-Stewart su \(A\).
  \end{block}
\end{frame}

\begin{frame}{Strategie}
  Fissiamo un gioco \(G(X, T)\).
  \begin{block}{Definizione}
     Una \textbf{strategia} per \(\mathrm{I}\) è un albero \(\sigma \subseteq T\) tale che
    \begin{itemize}
    \item[1.] \(\sigma\) è potato e non vuoto;
    \item[2.] se \((a_0, \ldots, a_{2j}) \in \sigma\) allora ogni \((a_0, \ldots, a_{2j}, a_{2j+1}) \in T\) è in \(\sigma\);
    \item[3.] se \((a_0, \ldots, a_{2j - 1}) \in \sigma\) allora esiste un unico \(a_{2j} \in A\) tale che \((a_0, \ldots, a_{2j - 1}, a_{2j}) \in \sigma\).
    \end{itemize}
  \end{block}
\end{frame}

\begin{frame}{Strategie}
  Se \(A = \{0, 1, 2\}\) e \(T = A^{<\omega}\) allora
  \begin{equation*}
    \begin{tikzpicture}
      \node (1) at (0, 0){\(0\)};
      \node (11) at (-2, -1){\(0\)};
      \node (12) at (0, -1){\(1\)};
      \node (13) at (2, -1){\(2\)};
      \node (21) at (-2, -2){\(0\)};
      \node (22) at (0, -2){\(1\)};
      \node (23) at (2, -2){\(2\)};
      \node (dots) at (0, -3.5){\(\vdots\)};
      \node (dots) at (-2, -3.5){\(\vdots\)};
      \node (dots) at (2, -3.5){\(\vdots\)};

      \node (a) at (3.5, 0){\(\mathrm{I}\)};
      \node (a) at (3.5, -1){\(\mathrm{II}\)};
      \node (a) at (3.5, -2){\(\mathrm{I}\)};

      \draw[-] (1) -- (11) -- (21);
      \draw[-] (1) -- (12) -- (22);
      \draw[-] (1) -- (13) -- (23);

      \draw[-] (21) -- (-2.5, -3);
      \draw[-] (21) -- (-2, -3);
      \draw[-] (21) -- (-1.5, -3);

      \draw[-] (22) -- (-0.5, -3);
      \draw[-] (22) -- (0, -3);
      \draw[-] (22) -- (0.5, -3);

      \draw[-] (23) -- (1.5, -3);
      \draw[-] (23) -- (2, -3);
      \draw[-] (23) -- (2.5, -3);
    \end{tikzpicture}
  \end{equation*}
  è una strategia per \(\mathrm{I}\).
\end{frame}

\begin{frame}{Strategie}
  \begin{block}{Definizione}
    Una strategia \(\sigma \subseteq T\) per \(\mathrm{I}\) è \textbf{vincente} se \([\sigma] \subseteq X\) i.e. se \(\mathrm{I}\) vince ogni partita giocata seguendo \(\sigma\).
  \end{block}
  \pause
  Similmente definiamo strategie per \(\mathrm{II}\).
  \pause
  \begin{block}{Remark}
    Siccome \(G(X, T)\) non può finire in un pareggio non è possibile che sia \(\mathrm{I}\) che \(\mathrm{II}\) abbiano una strategia vincente.
  \end{block}
\end{frame}

\begin{frame}{Determinatezza}
  \begin{block}{Definizione}
    Un gioco \(G(X, T)\), o solamente l'insieme \(X \subseteq T\), si dice \textbf{determinato} se uno dei due giocatori ha una strategia vincente.
  \end{block}
  \pause
  \begin{block}{Domande}
    \begin{itemize}
    \item I chiusi e gli aperti sono determinati?
    \item I Boreliani sono determinati?
    \item Gli analitici sono determinati?
    \end{itemize}
  \end{block}
\end{frame}
\end{document}